\documentclass[11pt,]{article}
\usepackage{lmodern}
\usepackage{amssymb,amsmath}
\usepackage{ifxetex,ifluatex}
\usepackage{fixltx2e} % provides \textsubscript
\ifnum 0\ifxetex 1\fi\ifluatex 1\fi=0 % if pdftex
  \usepackage[T1]{fontenc}
  \usepackage[utf8]{inputenc}
\else % if luatex or xelatex
  \ifxetex
    \usepackage{mathspec}
  \else
    \usepackage{fontspec}
  \fi
  \defaultfontfeatures{Ligatures=TeX,Scale=MatchLowercase}
\fi
% use upquote if available, for straight quotes in verbatim environments
\IfFileExists{upquote.sty}{\usepackage{upquote}}{}
% use microtype if available
\IfFileExists{microtype.sty}{%
\usepackage{microtype}
\UseMicrotypeSet[protrusion]{basicmath} % disable protrusion for tt fonts
}{}
\usepackage[margin = 1.5in]{geometry}
\usepackage{hyperref}
\PassOptionsToPackage{usenames,dvipsnames}{color} % color is loaded by hyperref
\hypersetup{unicode=true,
            pdftitle={ARMA GARCH Processes},
            pdfauthor={Abhinav Anand, IIMB},
            colorlinks=true,
            linkcolor=blue,
            citecolor=magenta,
            urlcolor=red,
            breaklinks=true}
\urlstyle{same}  % don't use monospace font for urls
\usepackage{color}
\usepackage{fancyvrb}
\newcommand{\VerbBar}{|}
\newcommand{\VERB}{\Verb[commandchars=\\\{\}]}
\DefineVerbatimEnvironment{Highlighting}{Verbatim}{commandchars=\\\{\}}
% Add ',fontsize=\small' for more characters per line
\usepackage{framed}
\definecolor{shadecolor}{RGB}{248,248,248}
\newenvironment{Shaded}{\begin{snugshade}}{\end{snugshade}}
\newcommand{\KeywordTok}[1]{\textcolor[rgb]{0.13,0.29,0.53}{\textbf{#1}}}
\newcommand{\DataTypeTok}[1]{\textcolor[rgb]{0.13,0.29,0.53}{#1}}
\newcommand{\DecValTok}[1]{\textcolor[rgb]{0.00,0.00,0.81}{#1}}
\newcommand{\BaseNTok}[1]{\textcolor[rgb]{0.00,0.00,0.81}{#1}}
\newcommand{\FloatTok}[1]{\textcolor[rgb]{0.00,0.00,0.81}{#1}}
\newcommand{\ConstantTok}[1]{\textcolor[rgb]{0.00,0.00,0.00}{#1}}
\newcommand{\CharTok}[1]{\textcolor[rgb]{0.31,0.60,0.02}{#1}}
\newcommand{\SpecialCharTok}[1]{\textcolor[rgb]{0.00,0.00,0.00}{#1}}
\newcommand{\StringTok}[1]{\textcolor[rgb]{0.31,0.60,0.02}{#1}}
\newcommand{\VerbatimStringTok}[1]{\textcolor[rgb]{0.31,0.60,0.02}{#1}}
\newcommand{\SpecialStringTok}[1]{\textcolor[rgb]{0.31,0.60,0.02}{#1}}
\newcommand{\ImportTok}[1]{#1}
\newcommand{\CommentTok}[1]{\textcolor[rgb]{0.56,0.35,0.01}{\textit{#1}}}
\newcommand{\DocumentationTok}[1]{\textcolor[rgb]{0.56,0.35,0.01}{\textbf{\textit{#1}}}}
\newcommand{\AnnotationTok}[1]{\textcolor[rgb]{0.56,0.35,0.01}{\textbf{\textit{#1}}}}
\newcommand{\CommentVarTok}[1]{\textcolor[rgb]{0.56,0.35,0.01}{\textbf{\textit{#1}}}}
\newcommand{\OtherTok}[1]{\textcolor[rgb]{0.56,0.35,0.01}{#1}}
\newcommand{\FunctionTok}[1]{\textcolor[rgb]{0.00,0.00,0.00}{#1}}
\newcommand{\VariableTok}[1]{\textcolor[rgb]{0.00,0.00,0.00}{#1}}
\newcommand{\ControlFlowTok}[1]{\textcolor[rgb]{0.13,0.29,0.53}{\textbf{#1}}}
\newcommand{\OperatorTok}[1]{\textcolor[rgb]{0.81,0.36,0.00}{\textbf{#1}}}
\newcommand{\BuiltInTok}[1]{#1}
\newcommand{\ExtensionTok}[1]{#1}
\newcommand{\PreprocessorTok}[1]{\textcolor[rgb]{0.56,0.35,0.01}{\textit{#1}}}
\newcommand{\AttributeTok}[1]{\textcolor[rgb]{0.77,0.63,0.00}{#1}}
\newcommand{\RegionMarkerTok}[1]{#1}
\newcommand{\InformationTok}[1]{\textcolor[rgb]{0.56,0.35,0.01}{\textbf{\textit{#1}}}}
\newcommand{\WarningTok}[1]{\textcolor[rgb]{0.56,0.35,0.01}{\textbf{\textit{#1}}}}
\newcommand{\AlertTok}[1]{\textcolor[rgb]{0.94,0.16,0.16}{#1}}
\newcommand{\ErrorTok}[1]{\textcolor[rgb]{0.64,0.00,0.00}{\textbf{#1}}}
\newcommand{\NormalTok}[1]{#1}
\usepackage{graphicx,grffile}
\makeatletter
\def\maxwidth{\ifdim\Gin@nat@width>\linewidth\linewidth\else\Gin@nat@width\fi}
\def\maxheight{\ifdim\Gin@nat@height>\textheight\textheight\else\Gin@nat@height\fi}
\makeatother
% Scale images if necessary, so that they will not overflow the page
% margins by default, and it is still possible to overwrite the defaults
% using explicit options in \includegraphics[width, height, ...]{}
\setkeys{Gin}{width=\maxwidth,height=\maxheight,keepaspectratio}
\IfFileExists{parskip.sty}{%
\usepackage{parskip}
}{% else
\setlength{\parindent}{0pt}
\setlength{\parskip}{6pt plus 2pt minus 1pt}
}
\setlength{\emergencystretch}{3em}  % prevent overfull lines
\providecommand{\tightlist}{%
  \setlength{\itemsep}{0pt}\setlength{\parskip}{0pt}}
\setcounter{secnumdepth}{0}
% Redefines (sub)paragraphs to behave more like sections
\ifx\paragraph\undefined\else
\let\oldparagraph\paragraph
\renewcommand{\paragraph}[1]{\oldparagraph{#1}\mbox{}}
\fi
\ifx\subparagraph\undefined\else
\let\oldsubparagraph\subparagraph
\renewcommand{\subparagraph}[1]{\oldsubparagraph{#1}\mbox{}}
\fi

%%% Use protect on footnotes to avoid problems with footnotes in titles
\let\rmarkdownfootnote\footnote%
\def\footnote{\protect\rmarkdownfootnote}

%%% Change title format to be more compact
\usepackage{titling}

% Create subtitle command for use in maketitle
\newcommand{\subtitle}[1]{
  \posttitle{
    \begin{center}\large#1\end{center}
    }
}

\setlength{\droptitle}{-2em}

  \title{ARMA GARCH Processes}
    \pretitle{\vspace{\droptitle}\centering\huge}
  \posttitle{\par}
    \author{Abhinav Anand, IIMB}
    \preauthor{\centering\large\emph}
  \postauthor{\par}
      \predate{\centering\large\emph}
  \postdate{\par}
    \date{2018/07/29}

\linespread{1.25}
\usepackage{amsmath}

\begin{document}
\maketitle

\section{The Autocorrelation Function
(ACF)}\label{the-autocorrelation-function-acf}

The correlation between random variables \(X_1, X_2\) is a measure of
their linear dependence and is defined as:
\[\rho_{12}:= \frac{\text{cov}(X_1,X_2)}{\sqrt{\text{var}(X_1)\text{var}(X_2)}}=\frac{\sigma_{12}}{\sigma_1\sigma_2}\]

It lies between -1 and 1 and for normal random variables \(\rho_{12}=0\)
implies that the variables are independent.

If we have a sample \(\{x_{1,t}, x_{2,t}\}_{t=1}^T\) the correlation can
be consistently estimated by computing sample correlation:
\[\hat{\rho}_{12}=\frac{\hat{\sigma}_{12}}{\hat{\sigma_1}\hat{\sigma_2}}\]

For a time series \(r_t\) which is weakly stationary, the lag-\(l\)
autocorrelation function is the correlation between \(r_t\) and
\(r_{t-l}\):

\[\rho_l=\frac{\sigma_{t,t-l}}{\sigma_t\sigma_{t-l}}=\frac{\sigma_{t,t-l}}{\sigma_t^2}
=\frac{\gamma_l}{\gamma_0}\]

This follows from weak stationarity:
\(\sigma^2_t=\sigma^2_{t-l}=\gamma_0\) and
\(\text{cov}(r_t,r_{t-l})=\gamma_l\).

We claim that there is no autocorrelation if \(\rho_l=0\)
\(\forall l>0\).

To estimate the autocorrelation function of lag (say) 1, we use its
sample counterpart:

\[\hat{\rho}_1=\frac{\sum_{t=2}^T (r_t-\bar{r})(r_{t-1}-\bar{r})}{\sum_{t=1}^T (r_t-\bar{r})^2}\]

In general for lag \(l\) we consistently estimate it as:
\[\hat{\rho}_l=\frac{\sum_{t=l+1}^T (r_t-\bar{r})(r_{t-1}-\bar{r})}{\sum_{t=1}^T (r_t-\bar{r})^2}\]

The statistic \(\hat{\rho_1},\hat{\rho_2},\hdots\) is the \emph{sample
autocorrelation function} of \(r_t\) and is key to capturing the linear
dependence nature of the time series in question.

\section{Autoregressive (AR)
Processes}\label{autoregressive-ar-processes}

Perhaps last period's returns may have some significant impact on the
value of the returns this period. If so, its lag-1 autocorrelation may
be useful for predicting the current period's value:

\[r_t = \phi_0+\phi_1r_{t-1}+u_t\]

where \(u_t\) is weakly stationary with mean 0 and variance
\(\sigma^2_u\). This is simply equivalent to a regression where
\(r_{t-1}\) is the explanatory or independent variable.

It's straightforward to check the conditional mean and variance of such
a process:

\[\mathbb{E}(r_t|r_{t-1}) = \phi_0+\phi_1r_{t-1}\]
\[\text{var}(r_t|r_{t-1}) = \sigma_u^2\]

And more generally there could be defined autoregressive processes of
order \(p\) (\(AR(p)\)):

\[r_t=\phi_0+\phi_1r_{t-1}+\hdots+\phi_pr_{t-p} + u_t\]

\subsection{AR(1) processes}\label{ar1-processes}

Is the AR(1) process \(r_t=\phi_0+\phi_1r_{t-1}+u_t\) weakly stationary?
This will imply that its unconditional mean and variance must be fixed
in time and lag-\(l\) covariance must depend only on the lag length
\(l\).

\[\mathbb{E}(r_t) = \phi_0 + \phi_1\mathbb{E}(r_{t-1})+\mathbb{E}(u_t)\]
\[\mathbb{E}(r_t) = \phi_0 + \phi_1\mu\]
\[\mu = \frac{\phi_0}{1-\phi_1}\]

This clearly implies that for the mean of an AR(1) process to exist,
\(\phi_1\neq 1\) and \(\phi_0=\mu\cdot(1-\phi_1)=\mu-\mu\phi_1\).

Hence a weakly stationary AR(1) process is:
\[r_t=\mu-\mu\phi_1+\phi_1r_{t-1}+u_t\]
\[r_t-\mu=(r_{t-1}-\mu)\phi_1+u_t\]
\[r_t-\mu=((r_{t-2}-\mu)\phi_2+u_{t-1})\phi_1+u_t\] \[\vdots\]
\[r_t-\mu = u_t + \phi_1u_{t-1}+\phi_1^2u_{t-2}+\hdots\]
\[r_t=\mu+\sum_{i=0}^\infty\phi_1^i\cdot u_{t-i}\]

Additionally,

\[\text{var}(r_t)=\phi_1^2\text{var}(r_{t-1})+\sigma^2_u\]

Since for weakly stationary AR(1) processes
\(\text{var}(r_t)=\text{var}(r_{t-1})=\gamma_0\) we have
\[\gamma_0 = \frac{\sigma_u^2}{1-\phi_1^2}\]

Weak stationarity immediately implies that \(\phi_1\in(-1,1)\).

Hence taken together, for an AR(1) process to be weakly stationary it is
necessary and sufficient that \(\phi_1\in[-1,1]\); and the canonical
AR(1) series can be written as: \[r_t=(1-\phi_1)\mu+\phi_1r_{t-1}+u_t\]

\subsubsection{Autocorrelation Function for AR(1)
processes}\label{autocorrelation-function-for-ar1-processes}

We can easily check that for positive lags \(l>0\), the lagged
covariance follows:

\[\gamma_l = \phi_1\gamma_{l-1}\]

Hence it follows that for the autocorrelation function
\(\rho_l = \phi_1\rho_{l-1}\); and becasue \(\rho_0=1\),
\(\rho_l = \phi_1^l\). This implies that the autocorrelation function of
an AR(1) series decays exponentially with rate \(\phi_1\) and starting
value 1. If \(phi_1<0\) the series alternates between positive and
negative terms.

\textbf{Illustration}

For example let's compute the sample autocorrelation function (ACF) for
the financial market indices.

\begin{Shaded}
\begin{Highlighting}[]
\KeywordTok{acf}\NormalTok{(ret_BSE, }\DataTypeTok{na.action =}\NormalTok{ na.pass)}
\end{Highlighting}
\end{Shaded}

\begin{center}\includegraphics{FMC_T4_PhD_ARMA_GARCH_files/figure-latex/sample_ACF-1} \end{center}

\begin{Shaded}
\begin{Highlighting}[]
\KeywordTok{acf}\NormalTok{(ret_sp500, }\DataTypeTok{na.action =}\NormalTok{ na.pass)}
\end{Highlighting}
\end{Shaded}

\begin{center}\includegraphics{FMC_T4_PhD_ARMA_GARCH_files/figure-latex/sample_ACF-2} \end{center}

\begin{Shaded}
\begin{Highlighting}[]
\KeywordTok{acf}\NormalTok{(ret_nikkei, }\DataTypeTok{na.action =}\NormalTok{ na.pass)}
\end{Highlighting}
\end{Shaded}

\begin{center}\includegraphics{FMC_T4_PhD_ARMA_GARCH_files/figure-latex/sample_ACF-3} \end{center}

What about log-returns?

\begin{Shaded}
\begin{Highlighting}[]
\NormalTok{logret_BSE <-}\StringTok{ }\KeywordTok{func_pr_to_logret}\NormalTok{(index_bse}\OperatorTok{$}\NormalTok{Close)}
\NormalTok{logret_SP <-}\StringTok{ }\KeywordTok{func_pr_to_logret}\NormalTok{(ind_sp500}\OperatorTok{$}\NormalTok{SP500)}
\NormalTok{logret_Nikkei <-}\StringTok{ }\KeywordTok{func_pr_to_logret}\NormalTok{(ind_nikkei}\OperatorTok{$}\NormalTok{NIKKEI225)}

\NormalTok{ACF_BSE <-}\StringTok{ }\KeywordTok{acf}\NormalTok{(logret_BSE, }\DataTypeTok{na.action =}\NormalTok{ na.pass)}
\end{Highlighting}
\end{Shaded}

\begin{center}\includegraphics{FMC_T4_PhD_ARMA_GARCH_files/figure-latex/sample_ACF_logret-1} \end{center}

\begin{Shaded}
\begin{Highlighting}[]
\KeywordTok{barplot}\NormalTok{(}\KeywordTok{head}\NormalTok{(ACF_BSE}\OperatorTok{$}\NormalTok{acf))}
\end{Highlighting}
\end{Shaded}

\begin{center}\includegraphics{FMC_T4_PhD_ARMA_GARCH_files/figure-latex/sample_ACF_logret-2} \end{center}

\begin{Shaded}
\begin{Highlighting}[]
\NormalTok{ACF_SP <-}\StringTok{ }\KeywordTok{acf}\NormalTok{(logret_SP, }\DataTypeTok{na.action =}\NormalTok{ na.pass)}
\end{Highlighting}
\end{Shaded}

\begin{center}\includegraphics{FMC_T4_PhD_ARMA_GARCH_files/figure-latex/sample_ACF_logret-3} \end{center}

\begin{Shaded}
\begin{Highlighting}[]
\KeywordTok{barplot}\NormalTok{(}\KeywordTok{head}\NormalTok{(ACF_SP}\OperatorTok{$}\NormalTok{acf))}
\end{Highlighting}
\end{Shaded}

\begin{center}\includegraphics{FMC_T4_PhD_ARMA_GARCH_files/figure-latex/sample_ACF_logret-4} \end{center}

\begin{Shaded}
\begin{Highlighting}[]
\NormalTok{ACF_Nikkei <-}\StringTok{ }\KeywordTok{acf}\NormalTok{(logret_Nikkei, }\DataTypeTok{na.action =}\NormalTok{ na.pass)}
\end{Highlighting}
\end{Shaded}

\begin{center}\includegraphics{FMC_T4_PhD_ARMA_GARCH_files/figure-latex/sample_ACF_logret-5} \end{center}

\begin{Shaded}
\begin{Highlighting}[]
\KeywordTok{barplot}\NormalTok{(}\KeywordTok{head}\NormalTok{(ACF_Nikkei}\OperatorTok{$}\NormalTok{acf))}
\end{Highlighting}
\end{Shaded}

\begin{center}\includegraphics{FMC_T4_PhD_ARMA_GARCH_files/figure-latex/sample_ACF_logret-6} \end{center}

\subsubsection{Partial Autocorrelation Functions
(PACF)}\label{partial-autocorrelation-functions-pacf}

Is there a way to know how many lags to include for an autoregressive
return series? This issue is solved via the usage of \emph{partial
autocorrelation functions} as shown below.

Consider the following sequences of AR processes:

\[r_t=\phi_{01}+ \phi_{11}r_{t-1}+u_{1t}\]
\[r_t=\phi_{02}+ \phi_{12}r_{t-1}+\phi_{22}r_{t-2}+u_{2t}\]
\[r_t=\phi_{03}+ \phi_{13}r_{t-1}+\phi_{23}r_{t-2}+\phi_{33}r_{t-3}+u_{3t}\]
\[r_t=\phi_{04}+ \phi_{14}r_{t-1}+\phi_{24}r_{t-2}+\phi_{34}r_{t-3}+\phi_{44}r_{t-4}+u_{4t}\]
\[\vdots\]

These models are merely multiple regressions and can be estimated via
the standard least squares method.

In these models, \(\hat{\phi}_{11}\) is called the lag 1 sample PACF of
\(r_t\), \(\hat{\phi}_{22}\) of the second equation is the lag 2 sample
PACF of \(r_t\) and so on. By construction, the lag 2
\(\hat{\phi}_{22}\) is the marginal contribution of \(r_{t-2}\) in
explaining \(r_t\) over the AR(1) model and so on. Hence if the
underlying model is say AR(\(p\)) then all sample PACFs
\(\hat{\phi}_{11},\hdots, \hat{\phi}_{pp}\) must be different from 0 but
all sample PACFs from then on: \(\hat{\phi}_{p+1,p+1}, \hdots=0\). This
property can be used to find the order \(p\).

Armed with this knowledge, let's compute the PACFs for the three
financial market indices:

\begin{Shaded}
\begin{Highlighting}[]
\KeywordTok{pacf}\NormalTok{(ret_BSE, }\DataTypeTok{na.action =}\NormalTok{ na.pass)}
\end{Highlighting}
\end{Shaded}

\begin{center}\includegraphics{FMC_T4_PhD_ARMA_GARCH_files/figure-latex/PACF-1} \end{center}

\begin{Shaded}
\begin{Highlighting}[]
\KeywordTok{pacf}\NormalTok{(ret_sp500, }\DataTypeTok{na.action =}\NormalTok{ na.pass)}
\end{Highlighting}
\end{Shaded}

\begin{center}\includegraphics{FMC_T4_PhD_ARMA_GARCH_files/figure-latex/PACF-2} \end{center}

\begin{Shaded}
\begin{Highlighting}[]
\KeywordTok{pacf}\NormalTok{(ret_nikkei, }\DataTypeTok{na.action =}\NormalTok{ na.pass)}
\end{Highlighting}
\end{Shaded}

\begin{center}\includegraphics{FMC_T4_PhD_ARMA_GARCH_files/figure-latex/PACF-3} \end{center}

\subsubsection{Information Criteria}\label{information-criteria}

Apart from PACF, another way to find the number of lags is the use of
likelihood based information criteria. Here we look at the two most
famous ones: the Akaike Information Criterion (AIC) and the Bayesian
Information Criterion (BIC).

\[\text{AIC}(l) = -\frac{2}{T}\cdot\ln(\text{likelihood})+\frac{2}{T}\cdot(\#\text{parameters})\]

Hence for a Gaussian AR(\(l\)),
\(AIC=\ln(\hat{\sigma}^2_{u,MLE})+2\frac{l}{T}\). The first term
measures the good ness of fit of the model while the second penalizes
the usage of parameters.

The Bayesian Information Criterion (BIC) uses a different penalty
function. For a Gaussian AR(\(l\)) is takes the following form:

\[BIC(l)=\ln(\hat{\sigma}^2_{u,MLE})+\ln(T)\cdot \frac{l}{T}\]

\section{Moving Average (MA)
Processes}\label{moving-average-ma-processes}

Consider an infinitely long autoregressive process:

\[r_t = \phi_0+\phi_1r_{t-1}+\phi_2r_{r-2}+\hdots+u_t\]

If this series is to be weakly stationary, the coefficients \(\phi_j\)
must decay sufficiently fast. One way to ensure this is to assume that
\(\phi_j=-\theta^j\) for some \(\theta\in(0,1)\).
\[r_t = \phi_0-\theta_1r_{t-1}-\theta_1^2r_{t-2}-\hdots+u_t\]
\[r_t+\sum_{j=1}^\infty \theta_1^jr_{t-j}=\phi_0+u_t\] The same form can
be written for \(r_{t-2}\):
\[r_{t-1}+\sum_{j=2}^\infty \theta_1^jr_{t-j}=\phi_0+u_{t-1}\]

\section*{References}\label{references}
\addcontentsline{toc}{section}{References}

\hypertarget{refs}{}
\hypertarget{ref-Jondeau_Poon_Rockinger:2007}{}
Jondeau, Eric, Ser-Huang Poon, and Michael Rockinger. 2007.
\emph{Financial Modeling Under Non-Gaussian Distributions}. Springer
Finance.

\hypertarget{ref-Tsay:2010}{}
Tsay, Ruey S. 2010. \emph{Analysis of Financial Time Series}. Third
Edition. John Wiley; Sons.


\end{document}
